\chapter*{INTRODUCCIÓN}
\addcontentsline{toc}{chapter}{INTRODUCCIÓN}

\setlength{\parindent}{0pt}      
\setlength{\parskip}{1em}

Actualmente la música es uno de los productos de entretenimiento con más consumidores en todo el mundo, y su crecimiento ha sido exponencial desde que el formato digital se consolidó en la industria. Plataformas de streaming como Spotify ampliaron el acceso a la música, logrando conectar a sectores que no podían permitirse el formato físico con sus canciones favoritas a un precio considerablemente más bajo. Este cambio transformó la relación de las personas con el arte musical, pasando de ser un lujo a convertirse en un consumo cotidiano.



Spotify ha sido una empresa clave en la digitalización de la música desde la década de 2010, sin embargo, con la consolidación del formato digital se volvió determinante para las empresas del mercado seguir innovando para mantenerse por encima de los competidores.
Por ello, Spotify se ha decantado por invertir en el desarrollo de herramientas que fomenten la preferencia de los usuarios por su plataforma como su favorita para el consumo musical diario.


Durante años los desarrolladores de diferentes empresas han invertido tiempo y recursos en la investigación de métodos eficientes de recomendación para sistemas con información masiva, sin embargo, se han encontrado con dificultades crecientes relacionadas al consumo de recursos computacionales y al tiempo estimado de respuesta que, al hablar de empresas multinacionales con millones de usuarios, debe ser signficativamente rápido.


Como resultado de estos esfuerzos, el equipo de desarrolladores de Spotify ha diseñado diversas herramientas que mejoran la recomendación de música basadas en algoritmos que emplean técnicas de minería de datos y machine learning, logrando minimizar los obstaculos computacionales inherentes al problema a resolver.

No obstante, a pesar de que Spotify ha desarrollado algoritmos eficientes de búsqueda de recomendaciones, persisten áreas de oportunidad a ser tomadas en cuenta. En este proyecto se identifica una de ellas aún no abordada por la plataforma principal ni por herramientas externas y se propone un sistema desarrollado en el entorno web que soluciona esta problemática a través de una propuesta innovadora y creativa, complementando las investigaciones actuales que Spotify ha logrado en los últimos años.

\newpage
\thispagestyle{plain}
\vspace*{0.2cm}

\section*{PLANTEAMIENTO DEL PROBLEMA}
\addcontentsline{toc}{section}{PLANTEAMIENTO DEL PROBLEMA}

En la actualidad, el uso de la plataforma de Spotify le permite al usuario establecer sesiones de escucha de las siguientes maneras:

\begin{enumerate}
    \item \textbf{Elegir manualmente las canciones a escuchar:} Esto implica elegir cada una de las canciones que serán reproducidas en el espacio de tiempo que el usuario decida siguiendo un orden establecido. Este caso de uso se ha exponenciado con el uso de \textit{Listas de Reproducción} \footnote{\textbf{Lista de Reproducción: } Colección ordenada de canciones que se pueden reproducir de forma consecutiva. Se almacenan para ser usadas en el futuro.}.

    \item \textbf{Elegir una canción base como la fuente de recomendaciones:} Sucede cuando el usuario elige una canción que no pertenece a una Lista de Reproducción previamente almacenada, provocando que Spotify ejecute el algoritmo de recomendación tomando de referencia la canción elegida por el usuario. Además, la \textit{Canción base} también es usada cuando la Lista de Reproducción termina de reproducirse totalmente, generando el mismo comportamiento.
\end{enumerate}

Esto indica una problemática emergente: El usuario tiene poco o nulo control de las recomendaciones que la plataforma brinda, teniendo que elegir una sesión de escucha de forma manual o cambiando la \textit{canción base} constantemente. Como resultado, en largas sesiones de escucha, el usuario no sabrá qué canciones elegir o, en el caso de usar el algoritmo de recomendación, es probable que se aburra de escuchar temas similares por largos periodos de tiempo.

Esta situación refleja un área de oportunidad en la que el usuario pueda mantener un control más fino sobre el funcionamiento del algoritmo de recomendación, permitiendo así el diseño de sesiones de escucha prolongadas con mayor diversidad. De esta manera, se plantea la necesidad de explorar métodos alternativos de recomendación que complementen los enfoques actuales empleados por Spotify.

\newpage
\thispagestyle{plain}
\vspace*{0.2cm}

\section*{PROPUESTA DE SOLUCIÓN}
\addcontentsline{toc}{section}{PROPUESTA DE SOLUCIÓN}

En el presente proyecto se plantea una solución creativa e innovadora que le brinda al usuario una capa extra de personalización y control del algoritmo de recomendación a través del concepto de \textit{Viajes Musicales}.

En esta tesis se introducen un conjunto de conceptos que servirán como base para el desarrollo del sistema. En particular, se propone el concepto de \textit{Viaje Musical}, que es definido como una sesión de escucha formada por un conjunto de canciones, de las cuales el usuario solo se preocupará por las que se denominan \textit{estaciones}. El concepto de estaciones es definido como  aquellas canciones que el usuario elige y que representan la guía que tomarán los algoritmos de recomendación para forjar un camino coherente entre las diferentes estaciones, creando así un viaje musical guiado por el usuario pero enriquecido por el sistema.

Con el objetivo de alcanzar dicho nivel de interacción se propone desarrollar un sistema web en donde el usuario pueda vincular su cuenta de Spotify, elegir su viaje musical y vincular la lista de canciones resultante a su cuenta de forma automática, simplificando al máximo la experiencia del usuario para lograr una herramienta cotidiana y fácil de usar.

Por esta razón se busca desarrollar un algoritmo de recomendación que se inspire en los desarrollados por la plataforma aprovechando los recursos públicos que la empresa facilita como la \textit{Spotify Web API} o el algoritmo de recomendación \textit{Voyager}.

En conclusión, la propuesta no solo busca mejorar las recomendaciones existentes, sino ofrecer al usuario una experiencia activa, flexible y narrativa en su consumo musical. De esta manera, los viajes musicales representan un puente entre el control total del usuario y la automatización de los algoritmos, creando una experiencia de escucha dinámica y personalizada que actualmente no existe en la plataforma.

\newpage
\thispagestyle{plain}
\vspace*{0.2cm}

\section*{JUSTIFICACIÓN}
\addcontentsline{toc}{section}{JUSTIFICACIÓN}



En la actualidad, los algoritmos de recomendación representan un elemento clave en la competitividad de los sistemas de entretenimiento digital, ya que permiten ofrecer experiencias personalizadas y mantener la fidelidad de los usuarios. Comprender a fondo las aportaciones previas en este ámbito es esencial para aprovecharlas de manera óptima y generar mejoras que impacten de forma directa en la satisfacción de los usuarios.

Dentro de las técnicas más utilizadas, los algoritmos de Búsqueda de Vecinos Aproximadamente más Cercanos constituyen una herramienta con múltiples aplicaciones, incluyendo el ámbito musical. En particular, el algoritmo \textit{Voyager} fue desarrollado con el objetivo de simplificar la implementación de este tipo de metodologías. Sin embargo, su uso resulta limitado cuando no se entiende a profundidad su funcionamiento, lo cual puede derivar en discrepancias entre los filtros aplicados y los resultados obtenidos. Por esta razón, este proyecto se centra en explicar de manera detallada tanto los fundamentos de la Búsqueda de Vecinos Cercanos como la lógica interna de \textit{Voyager}, con el fin de optimizar su aprovechamiento en sistemas de recomendación.

Además, se propone una nueva capa de \textit{Filtrado por características Cualitativas} que \textit{Voyager} por naturaleza no es capaz de soportar, añadiendo así una innovación técnica enfocada en búsquedas heurísticas adicionales previo al paso de recomendación.

Ambas aportaciones desarrolladas en el contexto musical le permitirán a los usuarios vincularse de forma diferente con las sesiones de música, dándoles un control adicional sobre un algoritmo que, actualmente, está totalmente cerrado al funcionamiento técnico. Este proyecto propone conectar al usuario con su algoritmo y explotar esa interacción de forma innovadora.

Adicionalmente, la implementación de un sistema basado en esta metodología no solo contribuye a fortalecer la investigación en el área, sino que también ofrece a nuevos desarrolladores una base sólida y bien documentada para el diseño de arquitecturas emergentes. De esta manera, el presente trabajo no solo atiende una necesidad técnica, sino que también abre la posibilidad de inspirar nuevas propuestas en el campo de los sistemas de recomendación de cualquier ámbito.


\newpage
\thispagestyle{plain}
\vspace*{0.1cm}

\section*{OBJETIVO GENERAL}
\addcontentsline{toc}{section}{OBJETIVO GENERAL}


Desarrollar un sistema web de recomendación musical  que le permita al usuario generar sesiones de escucha completas a partir de un conjunto representativo de canciones que servirá como base, las cuales permitirán que el sistema complete la lista mediante recomendaciones personalizadas, implementando algoritmos de búsqueda inteligentes.


\subsection*{OBJETIVOS ESPECÍFICOS}
\addcontentsline{toc}{subsection}{OBJETIVOS ESPECÍFICOS}

\begin{enumerate}
    \item Investigar la metodología de Búsqueda de Vecinos Aproximadamente más Cercanos para identificar las deficiencias de los métodos actuales en el entorno de sistemas de recomendación revisando literatura científica y técnica relevante.
    
     \item Implementar el algoritmo \textit{Voyager} con el fin de resolver de forma eficiente el problema de recomendación abordado, configurando sus propiedades de manera justificada y bien documentada.

    \item Crear un algoritmo de \textit{Generación de Muestra Significativa} añadiendo una capa de filtrado \textit{Cualitativo} que ayude con el problema de recomendación musical. 
    
    \item Desarrollar un sistema web dividido en capas que logre implementar los algoritmos de recomendación planteados y, además, obtenga información de la \textit{Spotify Web API} que será usada para las recomendaciones. 
    
    \item Utilizar la \textit{Spotify Web API} para facilitar al usuario la posibilidad de vincular su cuenta de Spotify de forma segura y alineada a los estándares de la plataforma.
    
    \item Desarrollar una interface de usuario llamativa que muestre el concepto de \textit{Viaje Musical} y permita obtener una sesión de escucha coherente basada en \textit{Estaciones} haciendo uso de tecnología web.

\end{enumerate}